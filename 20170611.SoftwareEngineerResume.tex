%!TEX TS-program = xelatex
%!TEX encoding = UTF-8 Unicode

\documentclass[a4paper, 10pt]{article}
\usepackage{enumitem} % for leftmargin of itemize
\usepackage{sectsty} % for section styling like header font size

% LAYOUT %
% Font
%% Set \header command
\makeatletter
\newcommand\header{\@setfontsize\Huge{14}{14}} % for my name
\makeatother

\sectionfont{\fontsize{12}{12}\selectfont}

% Margins
\usepackage[margin=1in]{geometry} % for page margin

% Indentation
\setlength\parindent{.5in}

% Page Numbering
\pagenumbering{gobble}

% Table
\setlength{\arrayrulewidth}{0mm}
\setlength{\tabcolsep}{1pt}


% DOCUMENT %
\begin{document}

\noindent
\textbf{\header Michael L. Otte}
\vspace{-3mm}

\noindent
\line(1,0){452}\\[2mm]
\begin{tabular}{ |p{2.9in}|p{3.75in}| }
{\sl Cell:} (831) 704-6563 & {\sl LinkedIn:} www.linkedin.com/in/michaelotte13 \\[0mm]
{\sl Email:} MichaelOtte1@gmail.com & {\sl GitHub:} www.github.com/motte \\[3mm]
\end{tabular}\\[-5mm]

\section*{SELECTED EXPERIENCES}
\vspace{-3mm}

{\bf Freenome}{, South San Francisco, CA}\hfill {\bf September 2015-Current \\}
{\sl CTO, VP of Product, Software Engineer - freenome.com} \newline
\indent I created several big data pipelines, alone and in a small team, that would process raw genomic data (raw read data), conduct automated or semi-automated feature selection, and input into various statistical learning methods (e.g. various perceptrons, random forest, and more).  With the uniquely challenging "big" dataset, I had the opportunity to implement a number of feature selectors on the genomic datasets in various AMQP systems (which I also designed and implemented) and format them into different types of matrices.

\indent I also developed a web-based genome browser (a visualization to analyze one or more genomic samples), which is often used for manual pattern recognition, analysis, and inter-sample comparison from vcf format files (essentially analytical/processed files of genetic variants).  For about a year, I worked with engineers at Google, Microsoft Azure, and AWS to implement feature suggestions on their platform and improve their big data and genomics/bioinformatics offerings (e.g. Google cloud functions, azure IAMs, Google genomics, etc.).  I did not code for them, but provided direct suggestions from our experience on their platforms, had in-person meetings with their chief compliance officer and head of Google Cloud, and eventually formalized a partnership with them.  In addition to software engineering and computational bioinformatics, I was involved in the development of tolerance and empathy in the workplace (a culture of "it's ok to mess up!").  I was also significantly involved in hiring, partnership/investor talks, clinical lab regulatory talks, technical/regulatory documentation systems, and technical/business presentations.  I spent much of my time collaborating and learning how to work with a highly interdisciplinary team with some of the best in their respective fields.  I am happy to talk in greater detail about any of these projects and the other projects that I worked on (but not mentioned here).\\[-6mm]
\begin{itemize}[leftmargin=5mm] 
\itemsep -2pt
	\item Helped grow the company from 4 to ~30 people and raise over \$70 million (from seed to Series A)
	\item Created, lead, and developed 3+ critical projects (using Python, React, HTML/CSS/JS, D3.js)
	\item Worked with "data scientists," computational biologists, and applied mathematicians to implement statistical models
\end{itemize}
                 
\noindent
{\bf Acorn}{, New York, NY - \sl Exited}\hfill {\bf May 2014-May 2015 \\}
{\sl COO, Lead Engineer, Co-Founder - acornapp.co} \\
\indent Coded a novel, patented location-acquisition tech for mobile.  Setup a highly scalable distributed system.  Coded a RWD web application that was rated over 90\% by Google Page Speed and Yahoo YSlow by utilizing image compression to JS minimization.  This reduced overall latency and page load speeds by over 80\%.  Also, created a mobile-server client API and an OAuth 2.0 Protocol API to enable more powerful and modular computations for the mobile app.  Management, fundraising, work flow, and culture-building experience in startup setting. \\[-6mm]
\begin{itemize}[leftmargin=5mm] 
\itemsep -2pt
	\item Helped develop two novel IPs in location acquisition tech
	\item Full Stack Web and Mobile Development with backend API server
	\item Built development team and established startup workflow
\end{itemize}
                 
\noindent
{\bf Dartmouth-Hitchcock Medical Clinic}{, Lebanon, NH}\hfill {\bf May 2011-August 2013 \\}
{\sl Researcher - geiselmed.dartmouth.edu/rigby/}\\[-6mm]
\begin{itemize}[leftmargin=5mm] 
\itemsep -2pt
	\item Studied the effects of rituximab (a.k.a. rituxan) in depletion of B cells via antibody-mediated cell shaving and complement dependent cell cytotoxicity
	\item Performed over 100 experiments through the entire process from experimental design to analysis and QA/QC
	\item Techniques included assay prep, PBMC and sera isolation from whole blood, cell carrying, and laser flow cytometry
	\item Resulting paper: {\sl Induction of interleukin-6 production by rituximab in human B cells.} doi: 10.1002/art.38798
\end{itemize}      
                 
\section*{RECENT DEVELOPMENT EXPERIENCE}
\vspace{-3mm}
Python, Javascript (including React, Node.js, Angular.js, jQuery), HTML5+CSS3, PostgreSQL, MySQL, Nginx + gunicorn + uWsgi, various shell scripting, PHP ($\sim$2010), Java ($\sim$2007)
                 
\section*{INTERESTS}
\vspace{-3mm}
I enjoy contributing to open/free source projects (e.g. Airflow), conducting biological and ethnographic research, rock climbing, surfing, singing, and volunteering.  I started coding in Hypercard when I was 10, and proceeds from my early freelance work was donated to charities (I recommend UNICEF).  I used to be an EMT (an "ambulance driver") and was the personal assistant to a former U.S. surgeon general.

\section*{EDUCATION}
\vspace{-3mm}
{\bf Dartmouth College}, Hanover, NH \hfill June 2013 \\[-7mm]
\begin{itemize}[leftmargin=5mm] 
\itemsep -1mm
	\item[] {\sl B.A.} {Socio-Cultural Anthropology (Honors Student w/ Research)}
	\item[] {\sl ND} {Biology, Concentration in Immunology}
	\item[] {\sl ND} {Pre-Medicine}
\end{itemize}
                
\end{document}